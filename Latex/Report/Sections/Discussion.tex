\subsection{Recommendation for further research}
    As in this project only one model is used, there is no 
    extensive research done on other models that could be 
    used for classification. For instance, a random forest 
    model or a neural network could also be possible models 
    to address the prepayment probabilities. These models 
    could be further researched to possibly obtain more accurate 
    results. 
    \\\\
    Besides different models which do simple classification, 
    also time dependent models could be considered (e.g.
    auto regressive models), to use this kind of models 
    however, more time based data about the client is 
    needed. One can think of personal information like age
    and income, but also information about the property over 
    time. As this is information that cannot be provided in 
    an open source data set, we were not able to work with 
    this information. However, a bank can have access to 
    this kind of information and may be able to optimize
    the models using this data.
    For data that is less accessible, the bank can also 
    make use of simulation to obtain estimates of client 
    specific information. One can think of estimating the 
    fico on a client level using copulas which depend
    on client information.
    \\\\
    Furthermore, as noted before, the modelling is done 
    with a selected sample of the available open-source 
    data set provided by FreddieMac. An analysis between 
    the sample set and the full data set should be done 
    to determine if the sample is representable for the 
    whole set. 
    \\\\ 
    As the LASSO seems not to give that good results for 
    selecting risk drivers, other methods of variable 
    selecting could be considered. Due to time constraint 
    it was not possible to include this into this report. 
    However, one could try to use other models which 
    address variable importance to do variable selection. 
    For example, a random forest could be used for this. 