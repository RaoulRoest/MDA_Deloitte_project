A mortgage is a common agreement between two parties: lender and 
loanee. The lender, mostly banks, lends money to the loanee. The 
money is secured with collateral and in case of default or fail to 
follow agreements, the collateral could be sold to obtain the 
money. In this agreement, a series contractual payment is used to 
repay the mortgage. The contractual payments consist of repayment 
of the mortgage and interest calculated over the outstanding debt. 
The interest can be divided into two groups, one considers a fixed 
interest rate and the other a non-fixed interest rate. The 
interest rate determines the profit of het lender. 
\\\\
A lender takes several risks such as a failing payment of the 
loanee or changing interest rates. Another risk is the loanee to 
prepay. Repaying is paying a part of the borrowed money before the 
agreed date of the scheduled payments. One speaks of partial 
prepayment when a part of the outstanding debt is prepaid. 
Likewise, a full prepayment is defined as prepaying the remaining 
debt. There is a possibility for the lender to charge extra costs 
for the prepayments. Both the interest and the scheduled payments 
form the income on the lender. If the loanee decides to prepay, 
the interest rate obtained becomes smaller and income decreases.  
On small scales this gives no problems, however, instead of one 
loanee, a bank can write mortgages to many loanees at once. This 
can result in big losses in comparison to the expected income.  
\\\\
As behaviour of clients can be predicted, extra funds could be 
kept separate to decrease risk without keeping unnecessary 
reserves. In this report there is looked at open-source data 
collected by FreddieMac and aimed to predict prepayments based on 
this data. The prediction is based on lender level. We want to 
predict prepayments in the next timestep for a given time. First, 
the data is analysed in Section \ref{section_data_analysis}. In 
which also the prepayment rule is derived. Followed by the model 
and performance criteria in Section \ref{section_model}. Next, the 
obtained results are shown in Section \ref{section_results}. Last, 
a discussion is formed, consisting of a conclusion, improvements 
and extensions of the model in Section \ref{section_discussion}.
\\\\
Special thanks to:\\
Dr. D. Kurowicka (TUDelft)\\
R. Tuinhof (Deloitte)\\
R. van der Leij (Deloitte)
