A mortgage is a legal agreement between two parties which include a loan and 
a security in real property to secure the loan. The party that is providing 
the loan is called the lender. The party that receives the loan called 
the borrower. 
The lender, for instance a bank, lends money to the borrower, in return 
the right of mortgage over the collateral (the real property) is given to 
the lender by the borrower.
The loan is secured with the real property and in case of default 
this collateral could be sold to payoff the outstanding debt. The time 
until maturity, at which the outstanding debt needs to be paid, is 
called the loan term. 
Over the loan a certain amount of interest needs to be paid. In general, 
mortgages can be divided into two categories. Fixed and variable 
mortgages. For fixed mortgages, the interest rate is fixed over 
(a predefined part of) the loan term. For variable mortgages the 
interest over the loan changes periodically (mostly on a yearly basis).
\\\\
Banks sell different sort of mortgages. 
One of the most common ones, is the annual mortgages.  
In the loan agreement, a contractual payment schedule is included. 
With these payments the outstanding debt is repaid. When the contractual
payments are constant over the loan term, the mortgage is called 
an annual mortgage. Each payment can therefore be split in a 
interest bearing and a non-interest bearing part. The non-interest 
bearing part of the outstanding loan is called unpaid principle 
balance (abbreviated as UPB). If one knows the originate UPB, 
the interest rate and the loan term, one can calculate the 
outstanding debt for a fixed annual mortgage.  
\\\\
Besides the risk default and the interest rate market risk, 
a specific risk for mortgages is the prepayment risk. A prepayment 
occurs if the borrower is paying more to repay the loan, then there 
was agreed upon in the contract. One speaks of partial 
prepayment when a part of the outstanding debt is prepaid. 
Likewise, a full prepayment is defined as prepaying the complete 
outstanding debt before the end of the loan term. In the agreement 
there is a possibility for the lender to charge extra costs for a prepayment. 
Both the interest and the scheduled payments 
form the income on the lender. If the borrower decides to prepay, 
the interest rate incomes obtained can become smaller than expected, which 
can have financial consequences for the bank due to the fixed loan terms
and its possible liabilities \cite{jacobs2005modelling}. 
%This needs some other words I think. In dutch 
% I would say "en de verplichtingen die hier uit voortvloeien"
% I do not know for sure if this is clear now.  
On small scales this gives no problems, however, instead of one 
mortgage loan, a bank has a whole portfolio of different loans. 
Many prepayments can therefore affect the liabilities of the bank. This 
can result in big losses in comparison to the expected income.  
\\\\
As the bank has to finance the portfolio of mortgages, it will 
be specifically interested in the prepayment risk over the portfolio.
By modelling the prepayment behavior of the borrowers, extra funds could be 
kept separate to decrease risk without keeping unnecessary 
reserves \cite{jacobs2005modelling}.
Besides the lender, there is a more general interest to be able 
to price the prepayment risk. As incorrect pricing of the prepayment risk 
in mortgages also have consequences on the mortgage security market, it also 
affects the interest rates \cite{Chinloy1989}.  
This report shows a way of modelling this prepayment behavior. 
Using open-source data collected by FreddieMac, the model aims to predict 
the probabilities of borrower to prepay. 
\\\\
To obtain a model for predicting the prepayments of the borrowers, 
different steps were taken.
First, the data is analysed in Section \ref{section_data_analysis}. In 
this Section the prepayment definition is made more formal. Using this 
definition, it is possible to classify the prepayments within the data set. 
As the prepayments can be distinguished, an analysis is done to obtain 
various drivers for the prepayment risk. Afterwards, the model is further 
explained. A general description of the requirements of the model are given in 
Section \ref{section_model}. In this report, the multinomial logistic regression 
model is used to model the prepayments on a borrowers basis. Also the criteria 
for addressing the performance of the model are discussed in this section.   
Next, the obtained results are shown in Section \ref{section_results}. Lastly, 
a discussion is given. In this discussion, a conclusion with respect to the 
prepayment modeling is given. Besides this conclusion, the report also addresses 
improvements, extensions and ideas for further research on the topic. 
This discussion is given in Section \ref{section_discussion}.
\\\\
% Maybe this should be in a small section previously to the introduction. 
As the report is a result of the narrow collaboration between the Technical 
University of Delft and the company Deloitte, the writers of the report would 
like to give a special thanks to:
\\\\
Dr. D. Kurowicka (Technical University Delft)  \\
R. Tuinhof (Deloitte)                           \\
R. van der Leij (Deloitte)
