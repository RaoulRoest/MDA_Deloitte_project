\section{Obtain monthly payments in annuity mortgage}
    The annuity mortgage is characterize by its equal payments. 
    Suppose payments $B_1, \ldots, B_n$, for these payments the relation
    \begin{equation}
        B_1 = \ldots = B_n 
    \end{equation}
    holds. Note that at $t_0$ there is no payments (as the mortgage is granted).
    Let us denote the interest at time $t$ as $r_t$. Since we look at fixed mortgages 
    the relation
    \begin{equation}
        r_1 = \ldots = r_n
    \end{equation}
    holds for the interest rates. 
    Let us assume that we are in a risk neutral market. When we valuate the loan in the risk 
    neutral market, the present value of the outstanding debt, should be equal to the loan
    and hence
    \begin{equation}
        L = B (1 + r)^{-t_1} + \ldots B (1 + r)^{-t_n}
    \end{equation}
    where $L$ is the granted loan. Using the geometric series, the value of each payment 
    $B$ can be determined: 
    \begin{equation}
        B\left[
            \displaystyle\sum_{j=1}^{n} (1 + r)^{-t_j}  
        \right] = 
        B\left[
            \displaystyle\sum_{j=0}^{n} (1 + r)^{-t_j} - 1  
        \right] = 
        B \left[
            \dfrac{1 - (1 + r)^{n+1}}{1 - (1 + r)} - 1
        \right] =
        B \left[
            \dfrac{1 - (1 + r)^{n+1}}{r}  - 1
        \right].
    \end{equation}
    From this the monthly payments $B$ can be determined: 
    \begin{equation}
        B = \dfrac{L}{
            \left(
                \dfrac{1 - (1 + r)^{n+1}}{r} - 1
            \right)
        }
    \end{equation}
    The term 
    \begin{equation}
        \dfrac{1 - (1 + r)^{n+1}}{r} - 1        
    \end{equation}
    will we call the monthly factor. Note that the interest rate given is a yearly rate. The 
    corresponding monthly rate can be calculated by: 
    \begin{equation}
        (1 + r)^{\frac{1}{12}}.
    \end{equation} 